#include <Wire.h>
#include <Adafruit_PWMServoDriver.h>Adafruit_PWMServoDriver pwm = Adafruit_PWMServoDriver();
#define SERVOMIN 150 // this is the 'minimum' pulse length count (out of 4096)
#define SERVOMAX 600 // this is the 'maximum' pulse length count (out of 4096)1,0
int currentAngle[4];
// Array to store the current angle of each servo
bool angleUpdated = false; 
// Flag to keep track of whether a new angle has been setvoid setup() 
{  
	Serial.begin(9600);  pwm.begin();  pwm.setPWMFreq(60);  
	// Analog servos run at ~60 Hz updates  for (int i = 0; i < 4; i++) 
	{    
		currentAngle[i] = 90; // Initialize all servo angles to 90 degrees 
	}
}
void loop() 
{  
	if (Serial.available() > 0) 
	{   
		// Read the incoming motor number    
		int motorNumber = Serial.parseInt();  
		// Read the incoming angle degree   
		int servoAngle = Serial.parseInt();    
		// Map the degree value to a PWM value   
		int pwmValue = map(servoAngle, 0, 180, SERVOMIN, SERVOMAX);    
		// Set the PWM value for the motor    
		pwm.setPWM(motorNumber-1, 0, pwmValue);    
		// Update the current angle of the servo   
		currentAngle[motorNumber-1] = servoAngle;   
		// Set the angle updated flag to true   
		angleUpdated = true;   
		// Add a delay to slow down the servo movement   
		delay(500); // delay time in milliseconds  
	}
	// Print out the current angle of each servo in the serial monitor
	// only if a new angle has been set 
	if (angleUpdated) 
	{    
		for (int i = 0; i < 4; i++) 
		{     
			Serial.print("Servo ");      
			Serial.print(i+1);      
			Serial.print(" angle: ");     
			Serial.println(currentAngle[i]);   
		}    
		// Reset the angle updated flag   
		angleUpdated = false; 
	}
}
